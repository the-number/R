% 20220721 (C) Gunter Liszewski -*- mode: tex; -*-
% permutation basics

\def\a{\left(\matrix}
\def\b{\cr}
\def\c{\right)}

\def\d{\footnote*{http://web.mit.edu/sp.268/www/rubik.pdf}}
\def\e{\bigskip}

Further to the example\d
$$(1 2 3)(2 3 1) = (1 3 2)\ order\ 3$$

\e
$(123456)*(123)=(231456)$ and $(123456)*(231)=(231456)$!
It looks like $(123)=(231)$. Yes, they are cycles and
therefore $(123)=(231)=(312)$.

\e
Order!
$(123456)*(123)=(231456)$ and $(231456)*(123)=(312456)$;
now the third application of the permutation $(123)$ gives
$(312456)*(123)=(123456)$, back to the initial, standard,
permutation. Three applications of $(123)$ return to the
initial state. The $order$ of $(123)$ is $3$.
$$\a{ 123456 \b
      231456 }\c,
  \a{ 231456 \b
      312456 }\c,
  \a{ 312456 \b
      123456 }\c.$$


\e
Going backwards!
$(123)$ the other way around is $(321)$.
A canonical cycle would have the smallest element first, and
$(321)$ written in its cononical form is $(132)$.

\e
Parity!
The permutation 3-cycle $(123)$ can also be written as a
two 2-cycle permutation $(12)(13)$, which shows that
the $parity$ of $(123)$ is $even$.

$$(123456)*(123)=(231456);$$
$$(123456)*(12) =(213456),$$
$$(213456)*(13) =(231456).$$
\bye
