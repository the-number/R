% 20220722 (C) Gunter Liszewski -*- mode: tex; -*-
% permutation basics

\def\a{\left(\matrix}
\def\b{\cr}
\def\c{\right)}

\def\d{\footnote*{http://web.mit.edu/sp.268/www/rubik.pdf}}
\def\f{\footnote*{https://www-cs-faculty.stanford.edu/\%7Eknuth/taocp.html}}
\def\e{\bigskip}
\def\R{{Rubik}}
\def\*{\circ}

% 595.28 841.89 a5: 419.53 595.28
\vsize=300pt % trial and likely error
\hsize=480pt % landscape to view on your screen
\footline={\tenrm Permutations\quad\dotfill\quad \folio}

\beginsection 1. Plain \TeX nology % and some Maths

\e
Creating all possible permutations.  TAoCP\f 1.2.5.
Two methods are given.  Numbers of permutation are evaluated.

\e
First, the group's binary operation is named $composition$.
Given two elements $\pi,\eta\in\R$, then
$\pi\*\eta\in\R$.

\e
$\pi=(a_1 a_2 ... a_n),$ and $\eta=(b_1 b_2 ... b_n)$, then
$\pi\*\eta\in\R$ and $\eta\*\pi\in\R.$

\bye
