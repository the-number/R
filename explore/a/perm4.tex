% 20220724 (C) Gunter Liszewski -*- mode: tex; -*-
% permutation basics

\def\a{\left(\matrix}
\def\b{\cr}
\def\c{\right)}

\def\d{\footnote*{The Mathematics of the Rubik’s Cube,
  http://web.mit.edu/sp.268/www/rubik.pdf}}
\def\f{\footnote*{TAoCP chapter 1.2.5,
  https://www-cs-faculty.stanford.edu/\%7Eknuth/taocp.html}}
\def\g{\footnote*{\TeX book, texbook.tex,
  https://www.ctan.org/tex-archive/systems/knuth/dist/tex}}
%
\def\e{\bigskip}
\def\R{\hbox{Woozy}}
\def\*{\circ}
%
% 595.28 841.89 a5: 419.53 595.28
\vsize=300pt % trial and likely error
\hsize=480pt % landscape to view on your screen

\footline={\tenrm Permutations\quad\dotfill\quad \folio}
\raggedright

\beginsection 1. Plain \TeX nology % and some Maths

\proclaim Theorem T. All things are not necessarily the same\g

\beginsection 2. Permutations

\e
TAoCP in chapter 1.2.5 gives two methods to generate
all permutations of a given ordered set.
Quantites of permutations are considered
with relevance to computing efficiencies.

\beginsection 3. The Wide-Awake example Group

\e
We re-think\d, re-word, and re-start
with a set of attributes, elements or objects,\break
$W=\{$ woozy, vacuous, sleepy, wide-awake $\}.$
These elements are used to generate all possible
arrangements $\eta$ which are orderd $n$-tuples with
$1\le n\le4.$
For example, $\eta=($ woozy, wide-awake $)$ is a $2$-tuple.
Now the set $\R$ is the set of all permutations that jumble
such elements like $\eta.$

\e
Let $(\R,\*,0,-)$ be the group with the set $\R$, a binary
operation $\*$, a neutral elment $0$, and
for each element $\pi\in\R$ there is an inverse element
$-\pi\in\R$ such that $\pi\*-\pi=0.$

\e
For now, here, we call this group's binary operation $composition$.
Given two elements
$\pi,\eta\in\R,$ then
$\pi\*\eta\in\R$ and $\eta\*\pi\in\R.$

% \e
% in a woozy state of progress
% $\pi=(a_1 a_2 ... a_n),$ and $\eta=(b_1 b_2 ... b_n)$, then
% $\pi\*\eta\in\R$ and $\eta\*\pi\in\R.$

\eject

\beginsection 4. Creating the Woozy set

\proclaim Theorem X. An ordered set of $n$ elements
has $n!$~arrangements.

This had a little consideration.  Here, we
convey our understanding of the Permutations and
Factorials section.\f

Given a set of objects $W=\{a_1,a_2,...,a_n\}$.
$P_n$ is the set of arrangements given $n$ objects
$a_1,...,a_n$ $\in W$, such as
$\{(a_1,a_2,...a_n),(a_2,a_1,...),...\}$. For example,
with $W=\{1,2,3\}$, we have
$$P_3=\{(123),(231),(312),(132),(321),(213)\}.$$

\e
Method 1, now, moves from $n=3$ to $n=4$ as follows.
For each element in $P_{n-1}=P_3$, place element
$a_n$ in each possible vacuous position to arrive at
$P_n=P_4$, that is
$$P_4=\{(a_na_1a_2a_3),
       (a_1a_na_2a_3),
       (a_1a_2a_na_3),
       (a_1a_2a_3a_n),
       ...,
       (a_na_2a_1a_3),
       (a_2a_na_1a_3),
       (a_2a_1a_na_3),
       (a_2a_1a_3a_n)\}$$

\beginsection 5. Accounting for these Arrangements

\e
Adding up all permutations that are so generated we have
$p_n$ the number of all elements in $P_n$ % to be
%$$p_n=\sum_{k=1}^np_k.$$.

\e
And again, after some re-view, we sense a need to re-word.
$P_{nn}$ is the set of permuted $n$-tuples, and $P_n$ is
the, probably bigger, set of all the $k$-tuples with
$k\in\{1,2,..,n\}.$. In other words, $P_n$
may mean different things, or sets of things. This also
applies to quantities that could be denoted like $p_{nk},$
and $p_{nn},$ and in case of our big wide-awake bean bag,
which we sum up to $p_n;$  probably.

\e
First, we started with $p_n=\sum_{k=1}^n k!$
to be the quantity
$p_n$ that accounts for all the elements of
arrangements in set $P_n$, 
with $p_k=k!$ for $1\le k\le n$.

However, on the back of some scrap paper, we jotted down
$\{(1),(2),(3),(4)\}$ and saw that $\{(2),(3),(4)\}$ are
not included in our sum, and
$\{(12),(21),(13),(31),(14),(41),$
$(23),(32),(24),(42),(34),(43)\}$ has $10$ 2-tuples unaccounted
for, etc.)

\e
So, for now, given that $p_{nk}=n(n-1)...(n-k+1)$\f,
combined with $p_n=\sum_{k=1}^n p_{nk}$,
we count the number of arrangements of $n$ objects to be
$p_n=\sum_{k=1}^n {n!\over{(n-k)!}}$
or some such like.
\eject
\beginsection 6. Making concrete Space

{% for the purpose of writing about Method 1, here
\def\i#1 {\noindent $\R_{#1}=\{$}%
\def\j{$\}$}%
\def\h#1 {$($ #1 $)$}%
\def\k#1 #2 {$($ #1, #2 $)$}%
\def\ee{\medskip}%
\e
We now look  at the set $W$ that we enumerated above and
apply method 1 to arrange things.

\e
Given $W$ as above, we have\break
\i41 \h woozy ,
          \h vacuus ,
          \h sleepy ,
          \h wide-awake \j

\ee
Then, taking one step at a time and applying method 1,
given the set

\i11 \h sleepy \j\ together with another element,
wide-awake $\in W,$
and we get

\i22 \k wide-awake sleepy ,
       \k sleepy wide-awake \j.
}% this is the end, for here
{% another beginning of a local thing, here
\def\ee{\medskip}%
\def\V#1#2#3 #4.{\noindent%
     $#1_{#2#3}=\{$ #4 $\}$}%
\def\W#1#2 #3.{\noindent%
     $P_{#1#2}=\hbox{Woozy}_{#1#2}=\{$ #3 $ \}$}%
\e Let's start counting now. We have
\ee
\W21 ( sleepy )., and another \V P21 ( wide-awake ).
\ee
\W22 ( sleepy, wide-awake ),
     ( wide-awake, sleepy )..
\e
To sum up we get

$p_2=(p_{21}+p_{21})+p_{22},$ with

$p_{21}=1,$ the count for each set of one 1-tuple, and

$p_{22}=2,$ the count for the one set of two 2-tuples that
we have created so far.

\e
Compare things with the calculations that we made earlier,

$p_{21}={2!\over(2-1+1)!}=1,$ and
$p_{22}={2!\over(2-2+1)!}=2.$ and $p_2=2*p_{21}+p_{22}$

$p_2=\sum_{k=1}^2{2!\over(2-k+1)!}$ which has two terms
and evaluates to $p_2={2\over2}+{2\over1}$, and it looks
like there is something wrong here!

\e
Let's take our result from section 5 and adjust.

\e
$p_n=\sum_{k=1}^n(n-k+1)*{n!\over(n-k+1)!}$ and since

$(n-k+1)=(n-k+1)*(n-k)*(n-k-1)*...*1$ we can simplify and have

$p_n=\sum_{k=1}^n{n!\over(n-k)!}$ which, for our state,

yields the following sum in two terms, (given that $0!=1$)

$p_2={2\over1}+{2\over1}=2+2=4,$ which agrees with our
permutations' making.

\e
And while we are here we set a solid base by calculating the
simple case for the set $P_1$ for which $p_1=1$, as we have
counted just now.

\e
$p_n=\sum_{k=1}^n{n!\over(n-k)!}$
with $n=1$ yields $p_1={1!\over(1-1)!}=1$
and confirms the basic case.

\e
So, does our formula hold its stepping up.
Assuming that $p_n=\sum_{k=1}^n{n!\over(n-k)!}$ is correct,
we need to find the terms to make $p_{n+1}$ from that.
(to be continued)
\hfil\eject

\beginsection 7. A few observations to count

\indent%
$p_{22}=p_{21}=2$

$p_{33}=p_{32}$

$p_{nk}=n(n-1)...(n-k+1)={n!\over(n-k)!}$

$1!=0!=1$ seems a reasonable cause, since
$(n-(n-1))!=1!=1$ and $(n-n)!=0!=1.$

$p_{32}=2p_{31}$ as $p_{31}=3$ and $p_{32}=6$ and it
appears that is on the same ground as the previous line
of reasoning; the nature of $x!$ being $x(x-1)...3*2*1$,
while with increasing $k$ the positive integer sequence
is a steady factor. (to be re-worded)

So, $p_{n2}=2p_{n1}$, $p_{n3}=3p_{n2}$,

$p_{n4}=4p_{n3}$, or some such like, if you get my drift.

(well, maybe not)
}
\hfil\eject
\beginsection 8. The Series of Sequences

{% following a sequence, adding things up
\def\i#1 {\noindent $\R_{#1}=\{$}%
\def\j{$\}$}%
\def\h#1.{$\{ \hbox{#1} \}$}%
\def\hh#1.{$\{$ #1 $\}$}%
\def\k#1 #2 {$($ #1, #2 $)$}%
\def\ee{\medskip}%

$$\halign{\indent ( # ), &( # ), &( # ), & ( # )\cr
${4!\over{(4-1)!}}$&${4!\over{(4-2)!}}$&%
${4!\over{(4-3)!}}$&${4!\over{(4-4)!}}$\cr}$$

$$\halign{\indent ( # ), &( # ), &( # ), & ( # )\cr
${24\over6}$&${24\over2}$&%
${24\over1!}$&${24\over{0!}}$\cr}$$

\e
\h ( sequence, series ), ( Folge, Reihe ).

\e
\h 1 2 3 4.

\e
\h ( woozy ), ( vacuous ), ( sleepy ), ( wide-awake ).

\e\h ( 1 ),  ( 2 ),  ( 3 ),  ( 4 ).
\e\h ( 12 ), ( 21 ),
     ( 13 ), ( 31 ),
     ( 14 ), ( 41 ),
             ( 23 ), ( 32 ),
             ( 24 ), ( 42 ),
                     ( 34 ), ( 43 ).

\e\hh ( 312 ), ( 132 ), ( 123 ),
      ( 321 ), ( 231 ), ( 213 ),
               ( 412 ), ( 142 ), ( 124 ),
               ( 421 ), ( 241 ), ( 214 ).

%\e\hh ( 4123 ), ( 1423 ), ( 1243 ), ( 1234 ),
%     ( 4321 ), ( 3421 ), ( 3241 ), ( 3214 ),
%                ( 4132 ), ( 1432 ), ( 1342 ), ( 1324 ),
%                ( 4231 ), ( 2431 ), ( 2341 ), ( 2314 ),
%      ( 3412 ), ( 4312 ), ( 4132 ), ( 4123 ),
%      ( 2143 ), \dots.
% reject ( 3214 ), ( 2314 ), ( 2134 ), ( 2143 ).

\e\hh 1234, 2341, 3412, 4123,
      4321, 3214, 2143, 1432,
            2134, 1342, 3421, 4213,
            4312, 3124, 1243, 2431,
                  1324, 3241, 2413, 4132,
                  4231, 2314, 3142, 1423.

}% end of section 8,here
\bye



