% 20220808 (C) Gunter Liszewski -*- mode: tex; -*-
% permutation basics

\def\a{\left(\matrix }
\def\b{\cr}
\def\c{\right)}

\def\d{\footnote*{The Mathematics of the Rubik’s Cube,
  http://web.mit.edu/sp.268/www/rubik.pdf}}
\def\f{\footnote*{TAoCP chapter 1.2.5,
  https://www-cs-faculty.stanford.edu/\%7Eknuth/taocp.html}}
\def\g{\footnote*{\TeX book, texbook.tex,
  https://www.ctan.org/tex-archive/systems/knuth/dist/tex}}
% https://books.google.co.uk/books?id=_n1vr0_RbXoC&lpg=PA786&pg=PA786#v=onepage&q&f=false
% https://en.wikipedia.org/wiki/List_of_Unicode_characters
% escaped #=\%23 (do not escape &=\%26) _=\%5f
\def\gg{\footnote*{https://books.google.co.uk/books?id=\%5fn1vr0\%5fRbXoC\&lpg=PA786\&pg=PA786}}
% Singmaster, David, Moral and mathematical lessons from
% a Rubik cube, New Scientist 23/30 Dec 1982, page 786
% https://en.wikipedia.org/wiki/Rubik's_Cube_group
%
\def\dd{\footnote*{https://github.com/the-number/R/blob/explore/0001/gnubik/explore/a/p/6/in-pictures.org}}
\def\e{\bigskip}
\def\R{\hbox{Woozy}}
\def\*{\circ}
%
% 595.28 841.89 a5: 419.53 595.28
\vsize=300pt % trial and likely error
\hsize=480pt % landscape to view on your screen

\footline={\tenrm Permutations\quad\dotfill\quad \folio}
\raggedright

\beginsection 1. Plain \TeX nology % and some Maths

\proclaim Theorem T. All things are not necessarily the same\g

\def\strutdepth{\dp\strutbox}
\def\marginalstar{\strut\vadjust{\kern-\strutdepth\specialstar}}
\def\specialstar{\vtop to \strutdepth{
  \baselineskip\strutdepth
  \vss\llap{* }\null}}

\beginsection 2. Permutations

\e
TAoCP in chapter 1.2.5 gives two methods to generate
all permutations of a given ordered set.
Quantities of permutations are considered
with relevance to computing efficiencies.
\beginsection 3. The Wide-Awake example Group

\e
We re-think\d, re-word, and re-start
with a set of attributes, elements or objects,\break
$W=\{$ woozy, vacuous, sleepy, wide-awake $\}.$
These elements are used to generate all possible
arrangements $\eta$ which are ordered $n$-tuples with
$1\le n\le4.$
For example, $\eta=($ woozy, wide-awake $)$ is a $2$-tuple.
Now the set $\R$ is the set of all permutations that jumble
such elements like $\eta.$

\e
Let $(\R,\*,0,-)$ be the group with the set $\R$, a binary
operation $\*$, a neutral element $0$, and
for each element $\pi\in\R$ there is an inverse element
$-\pi\in\R$ such that $\pi\*-\pi=0.$

\e
For now, here, we call this group's binary operation $composition$.
Given two elements
$\pi,\eta\in\R,$ then
$\pi\*\eta\in\R$ and $\eta\*\pi\in\R.$

% \e
% in a woozy state of progress
% $\pi=(a_1 a_2 ... a_n),$ and $\eta=(b_1 b_2 ... b_n)$, then
% $\pi\*\eta\in\R$ and $\eta\*\pi\in\R.$

\eject

\beginsection 4. Creating the Woozy set

\proclaim Theorem X. An ordered set of $n$ elements
has $n!$~arrangements.

This had a little consideration.  Here, we
convey our understanding of the Permutations and
Factorials section.\f

Given a set of objects $W=\{a_1,a_2,...,a_n\}$.
$P_n$ is the set of arrangements given $n$ objects
$a_1,...,a_n$ $\in W$, such as
$\{(a_1,a_2,...a_n),(a_2,a_1,...),...\}$. For example,
with $W=\{1,2,3\}$, we have
$$P_3=\{(123),(231),(312),(132),(321),(213)\}.$$

\e
Method 1, now, moves from $n=3$ to $n=4$ as follows.
For each element in $P_{n-1}=P_3$, place element
$a_n$ in each possible vacuous position to arrive at
$P_n=P_4$, that is
$$P_4=\{(a_na_1a_2a_3),
       (a_1a_na_2a_3),
       (a_1a_2a_na_3),
       (a_1a_2a_3a_n),
       ...,
       (a_na_2a_1a_3),
       (a_2a_na_1a_3),
       (a_2a_1a_na_3),
       (a_2a_1a_3a_n)\}$$

\beginsection 5. Accounting for these Arrangements

\e
Adding up all permutations that are so generated we have
$p_n$ the number of all elements in $P_n$ % to be
%$$p_n=\sum_{k=1}^np_k.$$.

\e
And again, after some re-view, we sense a need to re-word.
$P_{nn}$ is the set of permuted $n$-tuples, and $P_n$ is
the, probably bigger, set of all the $k$-tuples with
$k\in\{1,2,..,n\}.$. In other words, $P_n$
may mean different things, or sets of things. This also
applies to quantities that could be denoted like $p_{nk},$
and $p_{nn},$ and in case of our big wide-awake bean bag,
which we sum up to $p_n;$  probably.

\e
First, we started with $p_n=\sum_{k=1}^n k!$
to be the quantity
$p_n$ that accounts for all the elements of
arrangements in set $P_n$, 
with $p_k=k!$ for $1\le k\le n$.

However, on the back of some scrap paper, we jotted down
$\{(1),(2),(3),(4)\}$ and saw that $\{(2),(3),(4)\}$ are
not included in our sum, and
$\{(12),(21),(13),(31),(14),(41),$
$(23),(32),(24),(42),(34),(43)\}$ has $10$ 2-tuples unaccounted
for, etc.)

\e
So, for now, given that $p_{nk}=n(n-1)...(n-k+1)$\f,
combined with $p_n=\sum_{k=1}^n p_{nk}$,
we count the number of arrangements of $n$ objects to be
$p_n=\sum_{k=1}^n {n!\over{(n-k)!}}$
or some such like.
\eject
\beginsection 6. Making concrete Space

{% for the purpose of writing about Method 1, here
\def\i#1 {\noindent $\R_{#1}=\{$}%
\def\j{$\}$}%
\def\h#1 {$($ #1 $)$}%
\def\k#1 #2 {$($ #1, #2 $)$}%
\def\ee{\medskip}%
\e
We now look  at the set $W$ that we enumerated above and
apply method 1 to arrange things.

\e
Given $W$ as above, we have\break
\i41 \h woozy ,
          \h vacuous ,
          \h sleepy ,
          \h wide-awake \j

\ee
Then, taking one step at a time and applying method 1,
given the set

\i11 \h sleepy \j\ together with another element,
wide-awake $\in W,$
and we get

\i22 \k wide-awake sleepy ,
       \k sleepy wide-awake \j.
}% this is the end, for here
{% another beginning of a local thing, here
\def\ee{\medskip}%
\def\V#1#2#3 #4.{\noindent%
     $#1_{#2#3}=\{$ #4 $\}$}%
\def\W#1#2 #3.{\noindent%
     $P_{#1#2}=\hbox{Woozy}_{#1#2}=\{$ #3 $ \}$}%
\e Let's start counting now. We have
\ee
\W21 ( sleepy ), ( wide-awake ).
\ee
\W22 ( sleepy, wide-awake ),
     ( wide-awake, sleepy )..
\e
To sum up we get

$p_2=p_{21}+p_{22},$ with

$p_{21}=2,$ the count for the set of two 1-tuples, and

$p_{22}=2,$ the count for set set of two 2-tuples that
we have created so far.

\e
Compare things with the calculations that we made earlier,

$p_{21}={2!\over(2-1)!}=2,$ and
$p_{22}={2!\over(2-2)!}=2.$ and $p_2=p_{21}+p_{22}$

$p_2=\sum_{k=1}^2{2!\over(2-k)!}$ which has two terms
and evaluates to $p_2={2\over1!}+{2\over0!}$, and it looks
better (or is this just an illusion; however, $1!=0!=1$).

\e
Let's take our result from section 5 and adjust.

\e
$p_n=\sum_{k=1}^n(n-k+1)*{n!\over(n-k+1)!}$ and since

$(n-k+1)!=(n-k+1)*(n-k)*(n-k-1)*...*1,$ we may simplify and have

$p_n=\sum_{k=1}^n{n!\over(n-k)!}$ which, for our state,

yields the following sum in two terms, (given that $0!=1$)

$p_2={2\over1}+{2\over1}=2+2=4,$ which agrees with our
permutations' making.

\e
And while we are here we set a solid base by calculating the
simple case for the set $P_1$ for which $p_1=1$, as we have
counted just now.

\e
$p_n=\sum_{k=1}^n{n!\over(n-k)!}$
with $n=1$ yields $p_1={1!\over(1-1)!}=1$
and confirms the basic case.

\e
So, does our formula hold its stepping up.
Assuming that $p_n=\sum_{k=1}^n{n!\over(n-k)!}$ is correct,
we need to find the terms to make $p_{n+1}$ from that.
(to be continued)
\hfil\eject

\beginsection 7. A few observations to count

\indent%
$p_{22}=p_{21}=2$

$p_{33}=p_{32}$

$p_{nk}=n(n-1)...(n-k+1)={n!\over(n-k)!}$

$1!=0!=1$ seems a reasonable cause, since
$(n-(n-1))!=1!=1$ and $(n-n)!=0!=1.$

$p_{32}=2p_{31}$ as $p_{31}=3$ and $p_{32}=6$ and it
appears that is on the same ground as the previous line
of reasoning; the nature of $x!$ being $x(x-1)...3*2*1$,
while with increasing $k$ the positive integer sequence
is a steady factor. (to be re-worded)

So, $p_{n2}=2p_{n1}$, $p_{n3}=3p_{n2}$, and $p_{n4}=4p_{n3}$,
or some such like.
(Well, probably it is not a conjecture that will
turn out to be true.)
}
\hfil\eject
\beginsection 8. The Series of Sequences

{% following a sequence, adding things up
\def\i#1 {\noindent $\R_{#1}=\{$}%
\def\j{$\}$}%
\def\h#1.{$\{ \hbox{#1} \}$}%
\def\hh#1.{$\{$ #1 $\}$}%
\def\k#1 #2 {$($ #1, #2 $)$}%
\def\ee{\medskip}%
%
$$\halign{\indent ( # ), &( # ), &( # ), & ( # )\cr
${4!\over{(4-1)!}}$&${4!\over{(4-2)!}}$&%
${4!\over{(4-3)!}}$&${4!\over{(4-4)!}}$\cr
${24\over6}$&${24\over2}$&%
${24\over1!}$&${24\over{0!}}$\cr
$P_{41}=4$&$P_{42}=12$&$P_{43}=24$&$P_{44}=24$\cr}$$

%$$\halign{\indent ( # ), &( # ), &( # ), & ( # )\cr
%${24\over6}$&${24\over2}$&%
%${24\over1!}$&${24\over{0!}}$\cr}$$
%
\e
\h ( sequence, series ), ( Folge, Reihe ).

\e
\h 1 2 3 4.

\e
\h ( woozy ), ( vacuous ), ( sleepy ), ( wide-awake ).

\e\h ( 1 ),  ( 2 ),  ( 3 ),  ( 4 ).
\e\h ( 12 ), ( 21 ),
     ( 13 ), ( 31 ),
     ( 14 ), ( 41 ),
             ( 23 ), ( 32 ),
             ( 24 ), ( 42 ),
                     ( 34 ), ( 43 ).

\e\hh ( 312 ), ( 132 ), ( 123 ),
      ( 321 ), ( 231 ), ( 213 ),
               ( 412 ), ( 142 ), ( 124 ),
               ( 421 ), ( 241 ), ( 214 ).
\eject
We find a 3-tuple, take its inverse, then
cycle down. These are order 3 cycles. Then, we
find another 3-tuple that is not yet noted, and
go back to the first step, until all cycle routes
have been followed.

$$\def\h #1 #2 { #1 & #2&&&\cr}%
\def\hh #1 #2 {& #1 & #2&&\cr}%
\def\hhh #1 #2 {&& #1 & #2&\cr}%
\def\hhhh #1 #2 {&&& #1 & #2 \cr}%
\halign{\indent # & # & # & # & # \hfil\cr
\h    123 321
\h    231 213
\h    312 132
\hh       124 421
\hh       241 214
\hh       412 142
\hhh          134 431
\hhh          341 314
\hhh          413 143
\hhhh             234 432
\hhhh             342 324
\hhhh             234 243
}$$

\eject
We find four order three cycles and their respective
inverse things (we may call, arbitrarily, the first
in its row the $id$ of its cycle,
while the other jumbles in the row follow their leader.
Maybe $chief$ is a more appropriate designation than $id$.)

\def\specialstar{\vtop to\strutdepth{
 \baselineskip\strutdepth
 \vss\llap{3-cycles}\null}}%
\marginalstar%
{\def\h #1 #2 #3 { #1 & #2 & #3 &&\cr}%
\def\hh #1 #2 #3 {& #1 & #2 & #3 &\cr}%
\def\hhh #1 #2 #3 {&& #1 & #2 & #3\cr}%
\def\hhhh #1 #2 #3 {&&& #1 & #2 & #3\cr}%
$$\halign{\indent # & # & # & # & # & # \hfil\cr
\h    123 231 312
\h    321 213 132
\hh       124 241 412
\hh       421 214 142
\hhh          134 341 413
\hhh          431 314 143
\hhhh             234 342 423
\hhhh             432 324 243
}$$}
\eject
%\e\hh ( 4123 ), ( 1423 ), ( 1243 ), ( 1234 ),
%     ( 4321 ), ( 3421 ), ( 3241 ), ( 3214 ),
%                ( 4132 ), ( 1432 ), ( 1342 ), ( 1324 ),
%                ( 4231 ), ( 2431 ), ( 2341 ), ( 2314 ),
%      ( 3412 ), ( 4312 ), ( 4132 ), ( 4123 ),
%      ( 2143 ), \dots.
% reject ( 3214 ), ( 2314 ), ( 2134 ), ( 2143 ).

\e\hh 1234, 2341, 3412, 4123,
      4321, 3214, 2143, 1432,
            2134, 1342, 3421, 4213,
            4312, 3124, 1243, 2431,
                  1324, 3241, 2413, 4132,
                  4231, 2314, 3142, 1423.
}% end of section 8,here
\eject
{
Again, we find three 4-cycles. We reverse to form their
inverse. We follow the four permutation instances that
are specified by these
six order four canonical cycle permutations.

As an additional step, here, in order to give some
visual~clue as to the uniqueness of each 4-tuple,
we have ordered the said permutation~cycles to list
the canonical cycle~representation as the first element
of its row. (As a permutation, each element
in each of the six rows below
will permute the standard permutation, say the
4-tuple~1234, to the identical resulting arrangement.)

How significant is this?  We aim to find an understanding
of how we may abstract this in the context of group~theory.
Given the 24 permutations listed, we have only
six group elements. Each has four different representations.
For example, the group's elements $3412$ and $4123$ are
equal because the permutation $(1234)\*(3412)=(2341)$
is equal to $(1234)\*(4123)=(2341).$

{\def\h #1 #2 #3 #4 { #1 & #2 & #3 & #4 &&\cr}%
\def\hh #1 #2 #3 #4 {& #1 & #2 & #3 & #4 &\cr}%
\def\hhh #1 #2 #3 #4 {&& #1 & #2 & #3 & #4\cr}%
$$\halign{\indent # & # & # & # & # & # \hfil\cr
\h    1234 2341 3412 4123
\h    1432 4321 3214 2143
\hh        1342 3421 4213 2134
\hh        1243 2431 4312 3124
\hhh            1423 4231 2314 3142
\hhh            1324 3241 2413 4132
}$$}
\def\specialstar{\vtop to\strutdepth{
 \baselineskip\strutdepth
 \vss\llap{4-cycles}\null}}%
\marginalstar%
}% the end of the three four-cycles and their inverses
{\eject%  section nine start
\beginsection 9. A Group of Permutations

\e\noindent We fill the vacuous bag with four words to make a start.
Let $W$ be the set that we made earlier,
namely\break
$\{$ woozy, vacuous, sleepy, wide-awake $\}.$

Next we make four sets of permutations $P_{nk}$ with
$k\in\{k|1\le k\le n\},$

We have $P_{41}$
{\def\h #1 { ( #1 ) \cr}%
\def\hh #1 {& ( #1 ) \cr}%
$$\halign{\indent # & # \hfil\cr%
\h    woozy
\h    vacuous
\hh           sleepy
\hh           wide-awake
}$$}

We have $P_{42}$
{\def\h #1,#2.{( #1 ) & ( #2 ) &&\cr}%
\def\hh #1,#2.{&( #1 )&( #2 )&\cr}%
\def\hhh #1,#2.{&&( #1 )&( #2 )\cr}%
$$\halign{\indent # & # & # & #\hfil\cr%
\h    woozy vacuous,     vacuous woozy.
\h    woozy sleepy,      sleepy woozy.
\h    woozy wide-awake,  wide-awake woozy.
\hh         vacuous sleepy,     sleepy vacuous.
\hh         vacuous wide-awake, wide-awake vacuous.
\hhh                sleepy wide-awake,  wide-awake sleepy.      
}$$}
\eject
\vfil
We have $P_{43}$
{\def\h #1,#2,#3.{( #1 )&( #2 )&( #3 )\cr}%
$$\halign{# & # & #\hfil\cr%
\h    woozy vacuous sleepy, vacuous sleepy woozy,
        sleepy woozy vacuous.
\h    sleepy vacuous woozy, vacuous woozy sleepy,
        woozy sleepy vacuous.}$$%
$$\halign{\ \ \ # & # & #\hfil\cr%
\h    woozy vacuous wide-awake,  vacuous wide-awake woozy,
        wide-awake woozy vacuous.
\h    wide-awake vacuous woozy, vacuous woozy wide-awake,
        woozy wide-awake vacuous.}$$%
$$\halign{\ \ \ \ \ \ # & # & #\hfil\cr%
\h    woozy sleepy wide-awake,  sleepy wide-awake woozy,
        wide-awake woozy sleepy.
\h    wide-awake sleepy woozy, sleepy woozy wide-awake,
        woozy wide-awake sleepy.}$$
$$\halign{\ \ \ \ \ \ \ \ \ # & # & #\hfil\cr%
\h    vacuous sleepy wide-awake,  sleepy wide-awake vacuous,
        wide-awake vacuous sleepy.
\h    wide-awake sleepy vacuous, sleepy vacuous wide-awake,
        vacuous wide-awake sleepy.}$$}%
\vfill
\eject\vfill
We have $P_{44}$ (now in progress)
{\def\h #1,#2.{( #1 )&( #2 )\cr}%
\def\hh #1,#2.{\hbox to 16pt{}( #1 )&\hbox to 16pt{}( #2 )\cr}%
$$\halign{# & #\hfil\cr%
\h    woozy vacuous sleepy wide-awake,
       vacuous sleepy wide-awake woozy.
\hh      sleepy wide-awake woozy vacuous,
          wide-awake woozy vacuous sleepy.
\h    woozy wide-awake sleepy vacuous,
       wide-awake sleepy vacuous woozy.
\hh     sleepy vacuous woozy wide-awake,
         vacuous woozy wide-awake sleepy.}$$%
$$\halign{\hbox to14pt{} # & #\hfil\cr%
\h    woozy sleepy wide-awake vacuous,
       sleepy wide-awake vacuous woozy.
\hh     wide-awake vacuous woozy sleepy,
         vacuous woozy sleepy wide-awake.
\h    woozy vacuous wide-awake sleepy,
       sleepy woozy vacuous wide-awake. 
\hh     wide-awake sleepy woozy vacuous,
         vacuous wide-awake sleepy woozy.}$$%
$$\halign{\hbox to28pt{}# & #\hfil\cr%
\h    woozy wide-awake vacuous sleepy,
       sleepy woozy wide-awake vacuous.
\hh     vacuous sleepy woozy wide-awake,
         wide-awake vacuous sleepy woozy.
\h    woozy sleepy vacuous wide-awake,
       wide-awake woozy sleepy vacuous.
\hh     vacuous wide-awake woozy sleepy,
         sleepy vacuous wide-awake woozy.}$$%
}}% end of section nine
\eject
\beginsection 10. The example 2F 2R cycle and its abstraction

\def\R{\hbox{\bf R}}%
\e Spinning the cube.  Let $\pi\in S$ and $S\subseteq \R$
where $S=\{ F, R \}$. We apply the front face-rotation twice,
followed by two right-face rotations, that is
$\pi=F\*F\*R\*R$.

We just did that on a Rubik's cube app, and counted
6 instances until the initial permutation was reinstantiated.
That indicates that $\pi$ may be of order $6$
(that is $24$ quarter turns in total.)

After one $FFRR$ we may do $5$ more $\pi$ composites, that is
$FFRR FFRR FFRR FFRR FFRR,$ to revert the first move.

We could, of course, also have done $-\pi$, that is the inverse
permutation of $\pi$ (with $\pi\*-\pi=0$). $-\pi=-R\*-R\*-F\*-F,$
which with $-F=f$ and $-R=r$, may be denoted as $rrff.$

\e Now we will enumerate the three subsets,
$F\subseteq \R,$ $R\subseteq \R,$ and $F\cup R\subseteq \R.$
These subsets of permutations contribute to the abstraction of
the Rubik's cube's front face $F$ and right face $R.$  The
union of these sets $S=R\cup F$ aims to help the enumeration
of permutations which we may call (not all that seriously)
jumbles, ruffles, permutes, and sometimes but not always
rotations or composites.  These subsets together with their
operation $\*$ and their neutral element $0$ (maybe called $()$)
form subgroups of the Rubik group $\R.$

\e The front face is represented by set $F,$ and set $R$
is the right face of the cube. (There is already some
ambiguity creeping and ready to interfere.
There are nouns like set, permutation, arrangement, and face;
then we see verbs such as permute, arrange, or jumble,
ruffle, or just rotate by a quarter turn; or some such
like.)

\eject
Here is the front face
$F=\{$ FUL FU FUR FR FDR FD FDL FL FC $\},$
and the right face
$R=\{$~RUF RU RUB RB RDB RD RDF RF RC $\}.$

We also see some entanglements, namely
$\{$ ( FUR RUF ), ( FDR RDF ), ( FR RF ) $\}.$
For example, FUR is the cube of cubes on the
front face in set $F,$
while RUF is the same cube (maybe called cuby)
on the right face in set $R.$
The elements FUR and RUF are also part of the
permutation (here, the action of rotating the face)
$F,$ and $R$ respectively.

\e The rotation of the front face as a permutation cycle is
$F=($ FUL FUR FDR FDL $)\*($ FU FR FD FL~$),$
and the right face re-arrangement by a clockwise quarter-turn
is $R=($ RUF RUB RDB RDF $)\*($ RU RB RD RF $),$
while the entanglements of the front and the right are
already lurking and ready to intervene.

\e Now we look at the FFRR permutation again.
This composite jumble has two front face quarter-turns followed
by two 90 degree turns of the right face and it gives the
permutation $F\*F\*R\*R,$ or in other more detailed words
$($ ( FUL FDR ) ( FUR FDL ) ( FU FD ) ( FR FL ) $\*$
    ( RUF RDB ) ( RUB RDF ) ( RU RD ) ( RF RB ) $)$
\eject
One way to verify this permutation (verb) is to express its
effect as a two-line expression, first $F(F\*F)$, like so
$$F(F\*F)=\left(\matrix{%
 \hbox{FUL FU FUR FR FDR FD FDL FL}\cr
 \hbox{FDR FD FDL FL FUL FU FUR FR}}\right),$$

\e %
Or in a more verbose way the right~face 180 degree
turn might look like this:

$$R(R\*R)=\left(\matrix{%
 \hbox{RFU RU RUB RB RBD RD RDF RF RC}\cr
 \hbox{RFU RU RUB RB RBD RD RDF RF RC}}\right)$$
{\def\*(#1){\circ\hbox{(#1)}}%
$$\*(RFU RBD)\*(RUB RDF)\*(RU RD)\*(RF RB)\*(RC)=$$}
$$\left(\matrix{%
 \hbox{RFU RU RUB RB RBD RD RDF RF RC}\cr
 \hbox{RBD RD RFD RF RFU RU RBU RB RC}}\right).$$
 
\e(to be corrected, still)

\eject
\e We try to visualise a particular face-to-face entanglement
which could be expressed within the context of the
respective edge-cubes of the front and the right~face.

\e {\def\a{\hskip 3cm}%
\def\b{\hskip 24pt}%
\def\c#1 {{\bf #1 }}%
\vskip 12pt%
\halign{#&#&#&#&#&#\cr
        &FU&&           &RU\cr
\a  FL&&\c FR &\b\c RF &&RB\cr
        &FD&&           &RD\cr}

\e {\def\a#1 #2 {\hbox{#1 #2}}%
In terms of the set of permutation cycles the entanglement may read
$\{(\a FR RF ),(\a RF FR )\},$ and again we note that
$(\a FR RF )=(\a RF FR ).$}

\e The corner-cubes have a similar entanglement. (re-think, re-word)
{\def\a{\hskip 3cm}%
\def\b{\hskip 24pt}%
\def\c#1 {{\bf #1 }}%
\vskip 12pt%
\halign{#&#&#&#&#&#\cr
\a FLU&&\c FUR &\b\c RFU &&RUB\cr
     &Front&&&       Right\cr
\a FDL&&\c FRD &\b\c RDF &&RD\cr}}

}% end of example 2F 2R cycle
\eject
{\beginsection 11. Let's look at the example RUru

The move given in the example\d\ is
$($ RUru $)$ which is said to be a ``commutator.''
It is said to affect seven cubes of which two have
no part in the up~face. In our abstraction we denote
$\gamma\in G$ and $\gamma=(\hbox{ RUru }),$ with
$G=U\cup R$ and $G\subseteq R.$

Here, we start with the standard~permutation (noun)
$\sigma$ with the edge~cubes of the up~face singled out
with the expression
$\sigma_U^e=(\hbox{ UB UR UF UL }).$

We re-vise our view of the clockwise quarter turn
of the right face's edges, and therefore write
$R^e=( \hbox{ RU RF RD RB } ).$  We also appreciate, now,
that the right face's up-edge, RU, is the
same as the up face's right~edge, UR. In other words,
$\hbox{RU}=\hbox{UR}.$ (Sounds trivial, but we find this
important to keep in mind, for now.)

So, just do it! That is, let's do the first component of
or $\gamma$-move. Here it is, looking at two faces with
two expressions.
$R^2=R\*(\hbox{ RU RF RD RB }),$ giving
$R^2=( \hbox{ RF RU RB RD } ).$
Now, at the same time we have
$U^R=U\*(\hbox{ UR RF }),$ being aware that
$\hbox{UR}=\hbox{RU},$ and noting the effect that
this $R$-move has on the $U$-face which we now call
$U^R,$ and which is spelled out as
$( \hbox{ UB RF UF UL } ).$

\eject
The second component of $\gamma$ being the top~face
quarter turn, $U,$ and we may write
$U^{RU}=R\*U.$ The, admittedly verbose expression reads
$U_e^{RU}=U^R\*U,$ and then
$U_e^{R}\*U=(\hbox{ UB RF UF UL })\*U$ and we get
$U_e^{RU}=(\hbox{ UL UB RF UF }).$

We also see the entanglement that this re-arrangement
implies and we aim to express the effect is has on the $R$-face.
$R_e^R\*U^{RU}$ which shows the end~effect, a ruffle of
$(\hbox{ RF UB }),$ or some such like (for now unconfirmed.)
This makes the $R$-face to read
$R_e^{RU}=R^2\*U^{RU}$ and
$R_e^{RU}=(\hbox{ UB RU RB RD }),$ (still under re-view;
in particular our mis-conceptions, the ambiguities, and
the mis-representations are creeping all over; however
we will whittle things down just to pick up each one
of these crawlies and re-permute things to form a set
of cohesion; given time.)
\vfil\eject
Fast forward, pending completion, we have booted the Rubik
cube to do the computation $\gamma$ for us, and just note
it down in terms of edge~jumbles,
$$\gamma_R^e=(\hbox{ RF RB RD BU })$$
$$\gamma_U^e=(\hbox{ UR FR UF UL })$$

Given the standard permutation $\sigma$ with
$\sigma_U^e=(\hbox{ UB UR UF UL }),$ and
$\sigma_R^e=(\hbox{ RU RB RD RF }),$
this allows us to calculate the components
in terms of permutation cycles
$$\gamma_R^e=(\hbox{RU RF})\*(\hbox{RB})\*(\hbox{RD})\*(\hbox{RF UL}),$$ and
$$\gamma_U^e=(\hbox{UB})\*(\hbox{UR FR})\*(\hbox{UF})\*(\hbox{UL}).$$
}% end of section 11
\eject
{\beginsection 12. Direction of the permutation
which cycles one face

\def\a#1{\hbox{ #1 }}% one cube
\def\c{\hbox{ \  }}% vacuousness
\def\d{$\*$ (clock-wise 1/4 face-turn)}% permute
\def\b#1 #2 #3 #4.{% b=left 4-typle, like  1 2 3 4.
         &\a  #1 &&         &\a  #4 \cr
\c\c\c \a #4 &\c&\a #2 &\d\a #3 &\c&\a #1\cr
         &\a  #3 &&         &\a  #2 \cr}%
\e\halign{#&#&#%
         &#&#&#\cr% a=prefix
\b 1 2 3 4.}% clock-wise rotation of a face

\e{\def\bb#1 {\hbox{#1}}%
\def\d {$\*\hbox{(FU FL FD FR)}=$}%
\halign{#&#&#%
         &#&#&#\cr% a=prefix
\b \bb FU \bb FR \bb FD \bb FL .}}%

\e{\def\bb#1 {\hbox{#1}}%
\def\d {$\*\hbox{(1432)}=$}%
\halign{#&#&#%
         &#&#&#\cr% a=prefix
\b \bb FU$_1$ \bb FR$_2$ \bb FD$_3$ \bb FL$_4$ .}}%

\e{\def\bb#1 {\hbox{#1}}%
\def\d {$\*\hbox{(1432)}=$}%
\halign{#&#&#%
         &#&#&#\cr% a=prefix
\b \bb woozy$_1$ \bb vacuous$_2$ %
\bb sleepy$_3$ \bb wide-awake$_4$ .}}%

\vfil\eject
\e The move $F$ again, with its effect on five faces.

\e{\def\bb#1 {\hbox{#1}}%
\def\d {$\*\hbox{(1432)}=$}%
\halign{#&#&#%
         &#&#&#\cr% a=prefix
\b \bb woozy$_1$ \bb vacuous$_2$ %
\bb sleepy$_3$ \bb wide-awake$_4$ .}}%

{\def\bb#1 {\hbox{#1}}%
\def\d {$\*F=$}%
\halign{#&#&#%
         &#&#&#\cr% a=prefix
\b \bb FU$_1$ \bb FR$_2$ %
\bb FD$_3$ \bb FL$_4$ .}}%

\vfil\eject\e Now, we look at some \TeX nicalities\g\/
in order to show these ruffles.
We want to utilise the $\backslash$halign
control sequence to layout our cube-faces.
We also want to use \TeX~definitions
to ease the writing of this \TeX-text.
We aim for some
yield of visual clues as to the Rubik group
model for these permutations. (re-word)

Here is the basic idea. We draw, using \TeX~text, the
four edge cubes: Up $U_1,$ Right $R_2,$ Down $D_3,$
and Left $L_4.$
{{$$\halign{#&#&#\cr%
 & $U_1$\cr%
$L_4$ && $R_2$\cr%
& $D_3$\cr}$$}}

Then with the help of \TeX\/ and its $\backslash$halign
control sequence, we abstract the drawing of this
particular face.
{\def\a{$$\halign{##&##&##\cr%
 & $U_1$\cr%
$L_4$ && $R_2$\cr%
& $D_3$\cr}$$}
\a}

\vfil\eject
Now we parameterise the four edge cubes that we see
on this face.
{\def\a#1 #2 #3 #4.{$$\halign{##&##&##\cr%
 & $#1_1$\cr%
$#4_4$ && $#2_2$\cr%
& $#3_3$\cr}$$}
\a U R D L.

This allows us, now, to facilitate these abstraction
and apply things to our favourite set.
\a \hbox{woozy} \hbox{vacuous} \hbox{sleepy} \hbox{wide-awake}.

And we may even show a plain and clear view of
these face rotations, in our own perception,
by using the set $\{1,2,3,4\}$ as the Group's set.
\a 1 2 3 4.

Let's see a representation of our adjoining
faces. These are the four faces that adjoin this one.
This one is face $(1,2,3,4).$  We say
that the face in front of us is face $0$ and
substitution yields $(01,02,03,04).$

\a 01 03 04 05.

Our edge cube $01$ is adjoined to face $1.$ (language)
and face one sees its edge~cube $10$ to be adjoined to
us. Yes! we are face $0.$

\a 12 13 10 15.

And here we take a break to reflect, and to replenish
our writing finger non-fossil, no-no-nothing fuel.

\vfil\eject\e Another approach to get these faces to show themselves side~by~side.

{\a 01 03 04 05.}

Now, then! we backup a few steps and just move straight ahead.

\e
{% \vcenter, \hbox and \vbox
$\vcenter{\halign{#&#&#\cr
&$B_1$\cr
$L_4$&&$R_2$\cr
&$F_3$\cr}}$

$\vcenter{\halign{#&#&#\cr
&$U_1$\cr
$L_4$&&$R_2$\cr
&$D_3$\cr}}$
\qquad
$\vcenter{\halign{#&#&#\cr
&$U_1$\cr
$F_4$&&$B_2$\cr
&$D_3$\cr}}$
\qquad
$\vcenter{\halign{#&#&#\cr
&$U_1$\cr
$R_4$&&$L_2$\cr
&$D_3$\cr}}$
\qquad
$\vcenter{\halign{#&#&#\cr
&$U_1$\cr
$B_4$&&$F_2$\cr
&$D_3$\cr}}$

$\vcenter{\halign{#&#&#\cr
&$F_1$\cr
$L_4$&&$R_2$\cr
&$B_3$\cr}}$
}% end of vcenter

{% \vcenter, \hbox and \vbox
\def\edges#1 #2 #3 #4.{$\vcenter{\halign{##&##&##\cr
&{#1}$_1$\cr
{#4}$_4$&&{#2}$_2$\cr
&{#3}$_3$\cr}}$}

\e We show the edges as number pairs where the first digit of each two-digit number
gives the cube's face visible on this cube of cubes' face.  The second such
digit gives the adjoining face of the standard permutation $\sigma,$ where the
faces are: $0$ to represent the front face,
$1$ its back, $2$ to the left, $3$ on the right, $4$ faces up, and $5$ looks down.

\vfil\eject
We break down $\sigma$ and show just its edges, with

\smallskip$\sigma_F^e=(02,05,03,04){}_F^e$, $\sigma_B^e=(12,14,13,15){}_B^e$,

$\sigma_U^e=(21,25,20,23){}_U^e$, $\sigma_D^e=(30,35,31,32){}_D^e$,

$\sigma_L^e=(42,40,43,41){}_L^e$, and $\sigma_R^e=(52,51,53,50){}_R^e$.
{\sl(this is unchecked and unfinished)}

\e% standard permutaion of Rubik's cube, in our view
\edges 21 25 20 23.\par\thinspace
\edges 02 05 03 04.\qquad\edges 52 51 53 50.\qquad
\edges 12 14 13 15.\qquad\edges 42 40 43 41.\par\thinspace
\edges 30 35 31 32.}% end of standard permutation

\e How can we see that this resembles some kind of reality?
First, since this is the standard permutation we have
all the edge cubes show their face in one colour, the
colour of the front~face being $0$, for example.

Second, one edge joins two faces and therefore,
for example the front face $0$ and the right face $5$
have a shared edge-cube. This shared edge appears on
the front-face on the right {\sl front}$_2$, here denoted
as $05{}_2.$
On the adjoining face, the right one in our example,
this shared edge is written as $50{}_4$ being on the
left, {\sl right}$_4$, of the right Rubik's cube face.

Third, we spot an error, as we experience just how to go
about error spotting. The left face ($4$) has in its
{\sl down} position ({\sl left}$_3$) a cube $43_3;$
and this feels correct.

But which edge is shared by the down cube ($3$)
and the left cube ($4$)? Answer: there should be a cube
$34_?$ on the {\sl down} face. We see that $30_1$ joins
with {\sl front} ($0$) correctly, and we  deduce that $34_4$ must replace $32_4$.

{\def\edges#1 #2 #3 #4.{$\vcenter{\halign{##&##&##\cr
&{#1}$_1$\cr
{#4}$_4$&&{#2}$_2$\cr
&{#3}$_3$\cr}}$}
\e Again, with one correction to read $34_4.$

\e% standard permutaion, corrected
\edges 21 25 20 23.\par\thinspace
\edges 02 05 03 04.\qquad\edges 52 51 53 50.\qquad
\edges 12 14 13 15.\qquad\edges 42 40 43 41.\par\thinspace
\edges 30 35 31 34.

\e Thanks, yes we see another error and we substitute
$24_4$ for the erroneous $23_4.$
This error hints at another tell-tale sign.
Each face should have exactly two even and two odd edges,
as a consequence of how our faces are numbered.
{\sl(This applies to the standard permutation
$\sigma^e$ and of course, after a jumble we might
see flipped edges.)}

\e% standard permutaion, s/23_4/24_4/
\edges 21 25 20 24.\par\thinspace
\edges 02 05 03 04.\qquad\edges 52 51 53 50.\qquad
\edges 12 14 13 15.\qquad\edges 42 40 43 41.\par\thinspace
\edges 30 35 31 34.}% end of edges defined space
}}% end of section 12
\hfil\eject
\beginsection 13. The concrete move RUru

{\def\edges#1 #2 #3 #4.{$\vcenter{\halign{##&##&##\cr
&{#1}$_1$\cr
{#4}$_4$&&{#2}$_2$\cr
&{#3}$_3$\cr}}$}

We start with the representation of the
standard permutation $\sigma^e$ in the context of
edge cubelets.

\e
\edges 21 25 20 24.\par\thinspace
\edges 02 05 03 04.\qquad\edges 52 51 53 50.\qquad
\edges 12 14 13 15.\qquad\edges 42 40 43 41.\par\thinspace
\edges 30 35 31 34.

And we re-write in a somewhat narrative style (as we
perceive it so to be.)

\smallskip$\sigma_F^e=(02,05,03,04){}_F^e$,
$\sigma_B^e=(12,14,13,15){}_B^e$,

$\sigma_U^e=(21,25,20,24){}_U^e$,
$\sigma_D^e=(30,35,31,34){}_D^e$,

$\sigma_L^e=(42,40,43,41){}_L^e$,
and $\sigma_R^e=(52,51,53,50){}_R^e$.

\hfil\eject
Now we apply the move $R,$ which spells out to be
$(1432)_R^e$, if you know what this means.  We perceive
the meaning to be $R_1\to R_4$, $R_2\to R_1$,
$R_3\to R_2$, and $R_4\to R_3$, and we think this to
be a clockwise $1/4$ turn around the $x$ axis, that
is the axis from left to right. {\it(We admit that the
concept of direction such as left and right is a
woozy one; but it's not vacuous.)}

\smallskip
\edges 21 05 20 24.\par\thinspace
\edges 02 35 03 04.\qquad\edges 50 52 51 53.\qquad
\edges 12 14 13 25.\qquad\edges 42 40 43 41.\par\thinspace
\edges 30 15 31 34.

\smallskip
We note that with the move of the right facelets we
also see changes on the adjoining faces, namely
$F_2$ sees $05\to 35$, $B_4$ moves $15\to 25$,
$U_2$ substitutes $25\to 05$, and finally $D_2$
transposes its edge $35\to 15$. (to verify)

\smallskip
Whittling things down, we see things scattered around us,
such as

\e$\{$( set, coset ), ( Definitionsmenge, Wertemenge )$\}$,

$\{$ injective, surjective, bijective $\}$.

\vfil\eject
We observe the effects of this rotation, $R^e$ as it
produces five permutation cycles of order four, namely

\smallskip
$R_R^e=(50, 53, 51, 52)_R^e$

$R_F^e=(05, 35, 15, 25)_F^e$

$R_U^e=(15, 25, 05, 35)_U^e$

$R_B^e=(25, 05, 35, 15)_B^e$

$R_D^e=(35, 15, 25, 05)_R^e$

\e
From a different view we see how the two-faced facelet $05$
will cycle through its positions in five
different faces.

\smallskip$05_F^e=F_2^e$

$05_U^e=U_2^e$

$05_B^e=B_4^e$

$05_D^e=D_2^e$

$50_R^e=(R_e^1, R_e^4, R_e^3, R_e^2)$.

\e {\it(We also appreciate, from our position, how the
abstractions may stay just abstract until we actually
follow things through, and maybe
even consider their views in their own surroundings.)}

\e In summary, we have

$05^e=(F_2^e, D_2^e, B_2^e, U_2^e)^e$ and

$50_R^e=(R_e^1, R_e^4, R_e^3, R_e^2)$.
}% end of section 13
\hfil\eject
\beginsection 14. The group of three slices

% https://books.google.co.uk/books?id=_n1vr0_RbXoC&lpg=PA786&pg=PA786
So called slices are sandwiched between opposing faces.
The group, so generated from the set of moves\break
$\{$~2R2l, 2F2b, 2U2d $\}$
is said to have 12 elements\gg.

Maybe we need a reminder of the group abstraction, and
maybe we should convince ourselves that this is a
group, and maybe we should account for all the elements,
the completeness, the associativities of the compositions,
the identity element, and the existence of inverse elements.

\e First, we count the basic generating elements:
1. $FFbb$, 2. $RRll$, 3. $UUdd$.

Next, we compose with two basics:
4. $FFbb\*RRll$, 5. $FFbb\*UUdd$, 6. $RRll\*UUdd$

Last come compositions from three basic generating elements:

7. $(FFbb\*RRll)\*UUdd$,
8. $(RRll\*FFbb)\*UUdd$,
9. $(FFbb\*UUdd)\*RRll$,

10.$(UUdd\*FFbb)\*RRll$,
11.$(RRll\*UUdd)\*FFbb$,
12.$(UUdd\*RRll)\*FFbb$.

\e{\it(This is the first attempt of enumerating this group
and the result is unverified.)}

\e We have, just now, taken some pictures of this group and
explored what we want to take forward.  We will look at
a subgroup of the three slices subgroup of the Rubik group.
We will just consider the centre-cubelets, their
permutation cycles, including their orientation.
We will also try to find the different aspects, or
view~points that will help us to understand what happens
with this group, how we denote things, and how this
simple example is possibly abstracted.

\e We already have seen different view points of the cube,
such as the view from the outside, and the view from a
particular edge-cubelet.  In the case of the $S^+$
group of three slices with only centre-cubelets we attempt
to describe the twelve elements, as given above, from
a view of a corner~cube, FUR that is not a part of
the underlying set of cubelets (only centre~facelets, now),
of our observations.

We are observing, sitting on the FUR corner, as the
elementary permuters effect the rotational moves,
in 180 degree quantities.  Sitting where we are sitting,
we see the slices rotate (while in effect we, that is
the FUR$=025=250=520$ face that we sit on,
will rotate in pairs with
our opposite face.)

Let's spell out, from an outside view, the move
$S_F^+$, with $S_F^+=(FFbb)$.

{\def\centres#1 #2 #3 #4 #5 #6.{$\vcenter{\halign{##&##&##\cr
{}$_1${#3}&&{}$_2${#4}\cr
&{}${}_{#2}{#1}^+$\cr
{}$_4${#6}&&{}$_3${#5}\cr}}$}

We start with the representation of the
standard permutation $\sigma^+$ in the context of
centre cubelets and the observer's, our cubelet FUR.
Yes, we add the orientation of the centre facelet, its spin,
as a sub-prefix, one of $\{0,1,2,3\}$ for $0$ degrees,
$90$ degrees, $180$ degrees, or $270$ degrees, or
$0,1/4,1/2,$ and $3/4$ facelet turns.

\e $\sigma^+_{025}$, at the start is

\e
\centres 2 0 {} {} 250 {}.\par\thinspace
\centres 0 0 {} 025 {} {}.\qquad\centres 5 0 502 {} {} {}.\qquad
\centres 1 0 {} {} {} {}.\qquad\centres 4 0 {} {} {} {} .\par\thinspace
\centres 3 0 {} {} {} {}.

\vfill\eject
$\sigma^+_{025}\*F^+$, and the centre facelet of our front face spins
by a $1/4$-turn, and we, the observer, spin likewise

\e
\centres 2 0 {} {} {} {}.\par\thinspace
\centres 0 1 {} {} 025 {}.\qquad\centres 5 0 {} 250 {} {}.\qquad
\centres 1 0 {} {} {} {}.\qquad\centres 4 0 {} {} {} {} .\par\thinspace
\centres 3 0 {} 502 {} {}.

\e And again we apply $F^+$,

\e
\centres 2 0 {} {} {} {}.\par\thinspace
\centres 0 2 {} {} {} 025.\qquad\centres 5 0 {} {} {} {}.\qquad
\centres 1 0 {} {} {} {}.\qquad\centres 4 0 {} {} 250 {} .\par\thinspace
\centres 3 0  502 {} {} {}.

(The observer feels woozy---are we lost?)

\vfill\eject
\e Now an inverse $B$ face turn, in our group this is
denoted $b^+$, and note the anti-clockwise spin of
centre-facelet from ${}_01^+\to{}_31^+$

\e
\centres 2 0 {} {} {} {}.\par\thinspace
\centres 0 2 {} {} {} 025.\qquad\centres 5 0 {} {} {} {}.\qquad
\centres 1 3 {} {} {} {}.\qquad\centres 4 0 {} {} 250 {} .\par\thinspace
\centres 3 0  502 {} {} {}.

(The observer is down, left in front---resting vacuously.)

\e And another $b^+$ $1/4$ anti-clock-wise turn with its respective
spin of facelet ${}_31^+\to{}_21^+$

\e
\centres 2 0 {} {} {} {}.\par\thinspace
\centres 0 2 {} {} {} 025.\qquad\centres 5 0 {} {} {} {}.\qquad
\centres 1 2 {} {} {} {}.\qquad\centres 4 0 {} {} 250 {} .\par\thinspace
\centres 3 0  502 {} {} {}.

\e (The observer feels like having been taken for a ride---%
wooziness prevails)

}% end of centre cubelets
\vfil\eject
\beginsection 15. Taking a more visual approach to these cubelets

{% about our excursion into the POV-ray world
A visual approach to the mathematics of the Rubik cube has been started.\dd
The {\sl Persistence of Vision Ray Tracer} allows to specify objects, such as
the cubelets, and the cube in a programming language with an appropriate
vocabulary.  A fundamental connection between the mathematics of the cube and
of its\break
visual model is achieved with the help of the application {\sl POV-ray.}

This is a very basic use of the {\sl POV-ray language.}  We define a square polygon
for each of our six faces.  We have $0$, $1$ for front and back; $2$, $3$ are up and down;
and $4$ and $5$ face left and right. In other words these faces are $F$, $B$, $U$,
$D$,$L$, and $R$.

Then the six facelets are cubed together and a particle, a cublete is constructed.
With these particles, or cubelets, we assemble a $3x3x3$ cube that represents and
allows us to model the Rubik cube's permutation groups.

We have tried two approaches to modify the rotation of one of the Rubik cube's faces.
First, the obvious route is to define a graphic representation,
in POV-ray language, of the front face, the back face, together with the middle slice.
Given this model we can apply the $F$-permutation by simply rotating
the $F$ face of the cube.

\e A different approach, which may be more practical, is to rotate each respective
cubelet of the front face to achieve the same desired result, which is to rotate the
front face of the Rubik cube.

In terms of possible permutations, this consideration shows that it is obvious that
we have many more possible arrangements; for example, we may now have more than one
edge cube of the same colours.  This is, of course, not possible with the
real Rubik cube.

It also opens up more views onto the mathematical models of the cube.  For example,
we could devise a metric of the distance between different cube arrangements.
(re-think, re-word, re-write)

}% end of this excursion
\bye
