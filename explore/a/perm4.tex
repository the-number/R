% 20220726 (C) Gunter Liszewski -*- mode: tex; -*-
% permutation basics

\def\a{\left(\matrix}
\def\b{\cr}
\def\c{\right)}

\def\d{\footnote*{The Mathematics of the Rubik’s Cube,
  http://web.mit.edu/sp.268/www/rubik.pdf}}
\def\f{\footnote*{TAoCP chapter 1.2.5,
  https://www-cs-faculty.stanford.edu/\%7Eknuth/taocp.html}}
\def\g{\footnote*{\TeX book, texbook.tex,
  https://www.ctan.org/tex-archive/systems/knuth/dist/tex}}
%
\def\e{\bigskip}
\def\R{\hbox{Woozy}}
\def\*{\circ}
%
% 595.28 841.89 a5: 419.53 595.28
\vsize=300pt % trial and likely error
\hsize=480pt % landscape to view on your screen

\footline={\tenrm Permutations\quad\dotfill\quad \folio}
\raggedright

\beginsection 1. Plain \TeX nology % and some Maths

\proclaim Theorem T. All things are not necessarily the same\g

\def\strutdepth{\dp\strutbox}
\def\marginalstar{\strut\vadjust{\kern-\strutdepth\specialstar}}
\def\specialstar{\vtop to \strutdepth{
  \baselineskip\strutdepth
  \vss\llap{* }\null}}

\beginsection 2. Permutations

\e
TAoCP in chapter 1.2.5 gives two methods to generate
all permutations of a given ordered set.
Quantites of permutations are considered
with relevance to computing efficiencies.

\beginsection 3. The Wide-Awake example Group

\e
We re-think\d, re-word, and re-start
with a set of attributes, elements or objects,\break
$W=\{$ woozy, vacuous, sleepy, wide-awake $\}.$
These elements are used to generate all possible
arrangements $\eta$ which are orderd $n$-tuples with
$1\le n\le4.$
For example, $\eta=($ woozy, wide-awake $)$ is a $2$-tuple.
Now the set $\R$ is the set of all permutations that jumble
such elements like $\eta.$

\e
Let $(\R,\*,0,-)$ be the group with the set $\R$, a binary
operation $\*$, a neutral elment $0$, and
for each element $\pi\in\R$ there is an inverse element
$-\pi\in\R$ such that $\pi\*-\pi=0.$

\e
For now, here, we call this group's binary operation $composition$.
Given two elements
$\pi,\eta\in\R,$ then
$\pi\*\eta\in\R$ and $\eta\*\pi\in\R.$

% \e
% in a woozy state of progress
% $\pi=(a_1 a_2 ... a_n),$ and $\eta=(b_1 b_2 ... b_n)$, then
% $\pi\*\eta\in\R$ and $\eta\*\pi\in\R.$

\eject

\beginsection 4. Creating the Woozy set

\proclaim Theorem X. An ordered set of $n$ elements
has $n!$~arrangements.

This had a little consideration.  Here, we
convey our understanding of the Permutations and
Factorials section.\f

Given a set of objects $W=\{a_1,a_2,...,a_n\}$.
$P_n$ is the set of arrangements given $n$ objects
$a_1,...,a_n$ $\in W$, such as
$\{(a_1,a_2,...a_n),(a_2,a_1,...),...\}$. For example,
with $W=\{1,2,3\}$, we have
$$P_3=\{(123),(231),(312),(132),(321),(213)\}.$$

\e
Method 1, now, moves from $n=3$ to $n=4$ as follows.
For each element in $P_{n-1}=P_3$, place element
$a_n$ in each possible vacuous position to arrive at
$P_n=P_4$, that is
$$P_4=\{(a_na_1a_2a_3),
       (a_1a_na_2a_3),
       (a_1a_2a_na_3),
       (a_1a_2a_3a_n),
       ...,
       (a_na_2a_1a_3),
       (a_2a_na_1a_3),
       (a_2a_1a_na_3),
       (a_2a_1a_3a_n)\}$$

\beginsection 5. Accounting for these Arrangements

\e
Adding up all permutations that are so generated we have
$p_n$ the number of all elements in $P_n$ % to be
%$$p_n=\sum_{k=1}^np_k.$$.

\e
And again, after some re-view, we sense a need to re-word.
$P_{nn}$ is the set of permuted $n$-tuples, and $P_n$ is
the, probably bigger, set of all the $k$-tuples with
$k\in\{1,2,..,n\}.$. In other words, $P_n$
may mean different things, or sets of things. This also
applies to quantities that could be denoted like $p_{nk},$
and $p_{nn},$ and in case of our big wide-awake bean bag,
which we sum up to $p_n;$  probably.

\e
First, we started with $p_n=\sum_{k=1}^n k!$
to be the quantity
$p_n$ that accounts for all the elements of
arrangements in set $P_n$, 
with $p_k=k!$ for $1\le k\le n$.

However, on the back of some scrap paper, we jotted down
$\{(1),(2),(3),(4)\}$ and saw that $\{(2),(3),(4)\}$ are
not included in our sum, and
$\{(12),(21),(13),(31),(14),(41),$
$(23),(32),(24),(42),(34),(43)\}$ has $10$ 2-tuples unaccounted
for, etc.)

\e
So, for now, given that $p_{nk}=n(n-1)...(n-k+1)$\f,
combined with $p_n=\sum_{k=1}^n p_{nk}$,
we count the number of arrangements of $n$ objects to be
$p_n=\sum_{k=1}^n {n!\over{(n-k)!}}$
or some such like.
\eject
\beginsection 6. Making concrete Space

{% for the purpose of writing about Method 1, here
\def\i#1 {\noindent $\R_{#1}=\{$}%
\def\j{$\}$}%
\def\h#1 {$($ #1 $)$}%
\def\k#1 #2 {$($ #1, #2 $)$}%
\def\ee{\medskip}%
\e
We now look  at the set $W$ that we enumerated above and
apply method 1 to arrange things.

\e
Given $W$ as above, we have\break
\i41 \h woozy ,
          \h vacuus ,
          \h sleepy ,
          \h wide-awake \j

\ee
Then, taking one step at a time and applying method 1,
given the set

\i11 \h sleepy \j\ together with another element,
wide-awake $\in W,$
and we get

\i22 \k wide-awake sleepy ,
       \k sleepy wide-awake \j.
}% this is the end, for here
{% another beginning of a local thing, here
\def\ee{\medskip}%
\def\V#1#2#3 #4.{\noindent%
     $#1_{#2#3}=\{$ #4 $\}$}%
\def\W#1#2 #3.{\noindent%
     $P_{#1#2}=\hbox{Woozy}_{#1#2}=\{$ #3 $ \}$}%
\e Let's start counting now. We have
\ee
\W21 ( sleepy ), ( wide-awake ).
\ee
\W22 ( sleepy, wide-awake ),
     ( wide-awake, sleepy )..
\e
To sum up we get

$p_2=p_{21}+p_{22},$ with

$p_{21}=2,$ the count for the set of two 1-tuples, and

$p_{22}=2,$ the count for set set of two 2-tuples that
we have created so far.

\e
Compare things with the calculations that we made earlier,

$p_{21}={2!\over(2-1)!}=2,$ and
$p_{22}={2!\over(2-2)!}=2.$ and $p_2=p_{21}+p_{22}$

$p_2=\sum_{k=1}^2{2!\over(2-k)!}$ which has two terms
and evaluates to $p_2={2\over1!}+{2\over0!}$, and it looks
better (or is this just an illusion; however, $1!=0!=1$).

\e
Let's take our result from section 5 and adjust.

\e
$p_n=\sum_{k=1}^n(n-k+1)*{n!\over(n-k+1)!}$ and since

$(n-k+1)!=(n-k+1)*(n-k)*(n-k-1)*...*1,$ we may simplify and have

$p_n=\sum_{k=1}^n{n!\over(n-k)!}$ which, for our state,

yields the following sum in two terms, (given that $0!=1$)

$p_2={2\over1}+{2\over1}=2+2=4,$ which agrees with our
permutations' making.

\e
And while we are here we set a solid base by calculating the
simple case for the set $P_1$ for which $p_1=1$, as we have
counted just now.

\e
$p_n=\sum_{k=1}^n{n!\over(n-k)!}$
with $n=1$ yields $p_1={1!\over(1-1)!}=1$
and confirms the basic case.

\e
So, does our formula hold its stepping up.
Assuming that $p_n=\sum_{k=1}^n{n!\over(n-k)!}$ is correct,
we need to find the terms to make $p_{n+1}$ from that.
(to be continued)
\hfil\eject

\beginsection 7. A few observations to count

\indent%
$p_{22}=p_{21}=2$

$p_{33}=p_{32}$

$p_{nk}=n(n-1)...(n-k+1)={n!\over(n-k)!}$

$1!=0!=1$ seems a reasonable cause, since
$(n-(n-1))!=1!=1$ and $(n-n)!=0!=1.$

$p_{32}=2p_{31}$ as $p_{31}=3$ and $p_{32}=6$ and it
appears that is on the same ground as the previous line
of reasoning; the nature of $x!$ being $x(x-1)...3*2*1$,
while with increasing $k$ the positive integer sequence
is a steady factor. (to be re-worded)

So, $p_{n2}=2p_{n1}$, $p_{n3}=3p_{n2}$, and $p_{n4}=4p_{n3}$,
or some such like.
(Well, probably it is not a conjecture that will
turn out to be true.)
}
\hfil\eject
\beginsection 8. The Series of Sequences

{% following a sequence, adding things up
\def\i#1 {\noindent $\R_{#1}=\{$}%
\def\j{$\}$}%
\def\h#1.{$\{ \hbox{#1} \}$}%
\def\hh#1.{$\{$ #1 $\}$}%
\def\k#1 #2 {$($ #1, #2 $)$}%
\def\ee{\medskip}%
%
$$\halign{\indent ( # ), &( # ), &( # ), & ( # )\cr
${4!\over{(4-1)!}}$&${4!\over{(4-2)!}}$&%
${4!\over{(4-3)!}}$&${4!\over{(4-4)!}}$\cr
${24\over6}$&${24\over2}$&%
${24\over1!}$&${24\over{0!}}$\cr
$P_{41}=4$&$P_{42}=12$&$P_{43}=24$&$P_{44}=24$\cr}$$

%$$\halign{\indent ( # ), &( # ), &( # ), & ( # )\cr
%${24\over6}$&${24\over2}$&%
%${24\over1!}$&${24\over{0!}}$\cr}$$
%
\e
\h ( sequence, series ), ( Folge, Reihe ).

\e
\h 1 2 3 4.

\e
\h ( woozy ), ( vacuous ), ( sleepy ), ( wide-awake ).

\e\h ( 1 ),  ( 2 ),  ( 3 ),  ( 4 ).
\e\h ( 12 ), ( 21 ),
     ( 13 ), ( 31 ),
     ( 14 ), ( 41 ),
             ( 23 ), ( 32 ),
             ( 24 ), ( 42 ),
                     ( 34 ), ( 43 ).

\e\hh ( 312 ), ( 132 ), ( 123 ),
      ( 321 ), ( 231 ), ( 213 ),
               ( 412 ), ( 142 ), ( 124 ),
               ( 421 ), ( 241 ), ( 214 ).
\eject
We find a 3-tuple, take its inverse, then
cycle down. These are order 3 cycles. Then, we
find another 3-tuple that is not yet noted, and
go back to the first step, until all cycle routes
have been followed.

$$\def\h #1 #2 { #1 & #2&&&\cr}%
\def\hh #1 #2 {& #1 & #2&&\cr}%
\def\hhh #1 #2 {&& #1 & #2&\cr}%
\def\hhhh #1 #2 {&&& #1 & #2 \cr}%
\halign{\indent # & # & # & # & # \hfil\cr
\h    123 321
\h    231 213
\h    312 132
\hh       124 421
\hh       241 214
\hh       412 142
\hhh          134 431
\hhh          341 314
\hhh          413 143
\hhhh             234 432
\hhhh             342 324
\hhhh             234 243
}$$

\eject
We find four order three cycles and their respective
inverse things (we may call, arbitrarily, the first
in its row the $id$ of its cycle,
while the other jumbles in the row follow their leader.
Maybe $chief$ is a more appropriate designation than $id$.)

\def\specialstar{\vtop to\strutdepth{
 \baselineskip\strutdepth
 \vss\llap{3-cycles}\null}}%
\marginalstar%
{\def\h #1 #2 #3 { #1 & #2 & #3 &&\cr}%
\def\hh #1 #2 #3 {& #1 & #2 & #3 &\cr}%
\def\hhh #1 #2 #3 {&& #1 & #2 & #3\cr}%
\def\hhhh #1 #2 #3 {&&& #1 & #2 & #3\cr}%
$$\halign{\indent # & # & # & # & # & # \hfil\cr
\h    123 231 312
\h    321 213 132
\hh       124 241 412
\hh       421 214 142
\hhh          134 341 413
\hhh          431 314 143
\hhhh             234 342 423
\hhhh             432 324 243
}$$}
\eject
%\e\hh ( 4123 ), ( 1423 ), ( 1243 ), ( 1234 ),
%     ( 4321 ), ( 3421 ), ( 3241 ), ( 3214 ),
%                ( 4132 ), ( 1432 ), ( 1342 ), ( 1324 ),
%                ( 4231 ), ( 2431 ), ( 2341 ), ( 2314 ),
%      ( 3412 ), ( 4312 ), ( 4132 ), ( 4123 ),
%      ( 2143 ), \dots.
% reject ( 3214 ), ( 2314 ), ( 2134 ), ( 2143 ).

\e\hh 1234, 2341, 3412, 4123,
      4321, 3214, 2143, 1432,
            2134, 1342, 3421, 4213,
            4312, 3124, 1243, 2431,
                  1324, 3241, 2413, 4132,
                  4231, 2314, 3142, 1423.
}% end of section 8,here
\eject
{
Again, we find three 4-cycles. We reverse to form their
inverse. We follow the four permutation instances that
are specified by these
six order four canonical cycle permutations.

As an additional step, here, in order to give some
visual~clue as to the uniqueness of each 4-tuple,
we have ordered the said permutation~cyles to list
the canonical cycle~representation as the first element
of its row. (As a permutation, each element
in each of the six rows below
will permute the standard permutation, say the
4-tuple~1234, to the identical resulting arrangement.)

How significant is this?  We aim to find an understanding
of how we may abstract this in the context of group~theory.
Given the 24 permutations listed, we have only
six group elements. Each has four different representations.
For example, the group's elements $3412$ and $4123$ are
equal because the permutation $(1234)\*(3412)=(2341)$
is equal to $(1234)\*(4123)=(2341).$

{\def\h #1 #2 #3 #4 { #1 & #2 & #3 & #4 &&\cr}%
\def\hh #1 #2 #3 #4 {& #1 & #2 & #3 & #4 &\cr}%
\def\hhh #1 #2 #3 #4 {&& #1 & #2 & #3 & #4\cr}%
$$\halign{\indent # & # & # & # & # & # \hfil\cr
\h    1234 2341 3412 4123
\h    1432 4321 3214 2143
\hh        1342 3421 4213 2134
\hh        1243 2431 4312 3124
\hhh            1423 4231 2314 3142
\hhh            1324 3241 2413 4132
}$$}
\def\specialstar{\vtop to\strutdepth{
 \baselineskip\strutdepth
 \vss\llap{4-cycles}\null}}%
\marginalstar%
}% the end of the three four-cycles and their inverses
{\eject%  section nine start
\beginsection 9. A Group of Permutations

\e\noindent We fill the vacuous bag with four words to make a start.
Let $W$ be the set that we made earlier,
namely\break
$\{$ woozy, vacuous, sleepy, wide-awake $\}.$

Next we make four sets of permutaions $P_{nk}$ with
$k\in\{k|1\le k\le n\},$

We have $P_{41}$
{\def\h #1 { ( #1 ) \cr}%
\def\hh #1 {& ( #1 ) \cr}%
$$\halign{\indent # & # \hfil\cr%
\h    woozy
\h    vacuous
\hh           sleepy
\hh           wide-awake
}$$}

We have $P_{42}$
{\def\h #1,#2.{( #1 ) & ( #2 ) &&\cr}%
\def\hh #1,#2.{&( #1 )&( #2 )&\cr}%
\def\hhh #1,#2.{&&( #1 )&( #2 )\cr}%
$$\halign{\indent # & # & # & #\hfil\cr%
\h    woozy vacuous,     vacuous woozy.
\h    woozy sleepy,      sleepy woozy.
\h    woozy wide-awake,  wide-awake woozy.
\hh         vacuous sleepy,     sleepy vacuous.
\hh         vacuous wide-awake, wide-awake vacuous.
\hhh                sleepy wide-awake,  wide-awake sleepy.      
}$$}
\eject
\vfil
We have $P_{43}$
{\def\h #1,#2,#3.{( #1 )&( #2 )&( #3 )\cr}%
$$\halign{# & # & #\hfil\cr%
\h    woozy vacuous sleepy, vacuous sleepy woozy,
        sleepy woozy vacuous.
\h    sleepy vacuous woozy, vacuous woozy sleepy,
        woozy sleepy vacuous.}$$%
$$\halign{\ \ \ # & # & #\hfil\cr%
\h    woozy vacuous wide-awake,  vacuous wide-awake woozy,
        wide-awake woozy vacuous.
\h    wide-awake vacuous woozy, vacuous woozy wide-awake,
        woozy wide-awake vacuous.}$$%
$$\halign{\ \ \ \ \ \ # & # & #\hfil\cr%
\h    woozy sleepy wide-awake,  sleepy wide-awake woozy,
        wide-awake woozy sleepy.
\h    wide-awake sleepy woozy, sleepy woozy wide-awake,
        woozy wide-awake sleepy.}$$
$$\halign{\ \ \ \ \ \ \ \ \ # & # & #\hfil\cr%
\h    vacuous sleepy wide-awake,  sleepy wide-awake vacuous,
        wide-awake vacuous sleepy.
\h    wide-awake sleepy vacuous, sleepy vacuous wide-awake,
        vacuous wide-awake sleepy.}$$}%
\vfill
\eject\vfill
We have $P_{44}$ (now in progress)
{\def\h #1,#2.{( #1 )&( #2 )\cr}%
\def\hh #1,#2.{\hbox to 16pt{}( #1 )&\hbox to 16pt{}( #2 )\cr}%
$$\halign{# & #\hfil\cr%
\h    woozy vacuous sleepy wide-awake,
       vacuous sleepy wide-awake woozy.
\hh      sleepy wide-awake woozy vacuous,
          wide-awake woozy vacuos sleepy.
\h    woozy wide-awake sleepy vacuous,
       wide-awake sleepy vacuous woozy.
\hh     sleepy vacuous woozy wide-awake,
         vacuous woozy wide-awake sleepy.}$$%
$$\halign{\hbox to14pt{} # & #\hfil\cr%
\h    woozy sleepy wide-awake vacuous,
       sleepy wide-awake vacuous woozy.
\hh     wide-awake vacuous woozy sleepy,
         vacuous woozy sleepy wide-awake.
\h    woozy vacuous wide-awake sleepy,
       sleepy woozy vacuous wide-awake. 
\hh     wide-awake sleepy woozy vacuous,
         vacuous wide-awake sleepy woozy.}$$%
$$\halign{\hbox to28pt{}# & #\hfil\cr%
\h    woozy wide-awake vacuous sleepy,
       sleepy woozy wide-awake vacuous.
\hh     vacuous sleepy woozy wide-awake,
         wide-awake vacuous sleepy woozy.
\h    woozy sleepy vacuous wide-awake,
       wide-awake woozy sleepy vacuous.
\hh     vacuous wide-awake woozy sleepy,
         sleepy vacuous wide-awake woozy.}$$%
}% end of section nine
\eject
\beginsection 10. The example 2F 2R cycle and its abstraction

\def\R{\hbox{\bf R}}%
\e Spinning the cube.  Let $\pi\in S$ and $S\subseteq \R$
where $S=\{ F, R \}$. We apply the front face-rotation twice,
followed by two right-face rotations, that is
$\pi=F\*F\*R\*R$.

We just did that on a Rubik's cube app, and counted
6 instances until the initial permutation was reinstantiated.
That indicates that $\pi$ may be of order $6$
(that is $24$ quarter turns in total.)

After one $FFRR$ we may do $5$ more $\pi$ composites, that is
$FFRR FFRR FFRR FFRR FFRR,$ to revert the first move.

We could, of course, also have done $-\pi$, that is the inverse
permutation of $\pi$ (with $\pi\*-\pi=0$). $-\pi=-R\*-R\*-F\*-F,$
which with $-F=f$ and $-R=r$, may be denoted as $rrff.$

\e Now we will enumerate the three subsets,
$F\subseteq \R,$ $R\subseteq \R,$ and $F\cup R\subseteq \R.$
These subsets of permuations contribute to the abstraction of
the Rubik's cube's front face $F$ and right face $R.$  The
union of these sets $S=R\cup F$ aims to help the enumertion
of permutations which we may call (not all that seriously)
jumbles, ruffles, permutes, and sometimes but not always
rotations or composites.  These subsets together with their
operation $\*$ and their neutral element $0$ (maybe called $()$)
form subroups of the Rubik group $\R.$

}% end of example 2F 2R cycle
\bye



