% 20220722 (C) Gunter Liszewski -*- mode: tex; -*-
% permutation basics

\def\a{\left(\matrix}
\def\b{\cr}
\def\c{\right)}

\def\d{\footnote*{http://web.mit.edu/sp.268/www/rubik.pdf}}
\def\f{\footnote*{https://www-cs-faculty.stanford.edu/\%7Eknuth/taocp.html}}
\def\e{\bigskip}
\def\R{{Woozy}}
\def\*{\circ}

% 595.28 841.89 a5: 419.53 595.28
\vsize=300pt % trial and likely error
\hsize=480pt % landscape to view on your screen
\footline={\tenrm Permutations\quad\dotfill\quad \folio}

\beginsection 1. Plain \TeX nology % and some Maths

\proclaim Theorem T. All things are not necessarily the same

\beginsection 2. Permutations

\e
TAoCP\f~1.2.5 gives two methods to generate all permutations
of a given ordered set. Quantites of permutations are considered
with relevance to computing efficiencies.

\beginsection 3. The Wide-Awake example Group

\e
Let $(\R,\*,0,-)$ be the group with a set
$\R=\{woozy, vacuous, sleepy, wide-awake\}$, a binary
operation $\*$, a neutral elment $0$, and
for each element $\pi\in\R$ there is an inverse element
$-\pi\in\R$ such that $\pi\*-\pi=0$.

\e
For now, here, we call this group's binary operation $composition$.
Given two elements
$\pi,\eta\in\R,$ then
$\pi\*\eta\in\R$ and $\eta\*\pi\in\R.$

% \e
% in a woozy state of progress
% $\pi=(a_1 a_2 ... a_n),$ and $\eta=(b_1 b_2 ... b_n)$, then
% $\pi\*\eta\in\R$ and $\eta\*\pi\in\R.$

\bye
